Οι φοιτητές που έχουν ασχοληθεί με πειραματικό αντικείμενο, στο παρόν
κεφάλαιο μπορούν να περιγράψουν την πειραματική διάταξη, την τεχνική
του πειράματος ή τη συλλογή των πειραματικών τους δεδομένων.

Είναι στην κρίση του συγγραφέα της εργασίας αν θα κρατήσει το κεφάλαιο
ενιαίο ή θα το επιμερίσει στα δύο βασικά τμήματα που αφορούν τη
διάταξη και την ανάλυση των πειραματικών δεδομένων. Η συνεννόηση
με τον επιβλέποντα είναι πάντα αυτή που θα κρίνει την τελική διάταξη
των κεφαλαίων.

Συστήνεται η προσεκτική χρήση γραφημάτων και πινάκων που
συνοδεύουν την πειραματική διαδικασία. Για το λόγο αυτό, ακολουθούν
μερικά παραδείγματα χρήσης αυτών των στοιχείων, τα
οποία μπορεί να συμβουλευθεί ο αρχάριος στο \LaTeX, μαζί με
τις εκτενέστερες βιβλιογραφικές πηγές που υπάρχουν διάθεσιμες.

Επίσης, στο κείμενο υπάρχουν αναφορές στη βιβλιογραφία όπως συνήθως
παρουσιάζονται στο διεθνή επιστημονικό τύπο,
π.χ. οι \cite{2012_PM,2011_AIP_Cl44}. 