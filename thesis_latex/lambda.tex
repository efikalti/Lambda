\chapter{Αρχιτεκτονική Lambda}
Ένα από τα πιο σύνθετα προβλήματα που χρειάζεται να αντιμετωπίζουν τα Big Data συστήματα είναι η εύρεση λύσης/απάντησης σε πραγματικό χρόνο.Κατά την αρχική τους σχεδίαση, δεν υπήρχε καμία πρόβλεψη για την αντιμετώπιση αυτού του είδους των προβλημάτων και φαινόταν έξω από την σφαίρα των δυνατοτήτων τους. Με την υποστήριξη και την ανάπτυξη που έλαβαν τα χρόνια μετά την εμφάνιση τους, βοήθησαν να αναπτυχθούν και να εδραιωθούν σε όλο και περισσότερους τομείς της πληροφορικής. Ο τομέας των real-time analytics αποδείχθηκε ένα αρκετά μεγάλο πρόβλημα, λόγω του όγκου της πληροφορίας που αποθηκεύεται σε ένα τέτοιο σύστημα. Η λύση ακόμα δεν έχει δοθεί από ένα μόνο συγκεκριμένο σύστημα αλλά έχει περιγραφτεί μια αρχιτεκτονική συστημάτων που παράγει το σωστό αποτέλεσμα, με μερικούς περιορισμούς.Η αρχιτεκτονική αυτή ονομάζεται Lambda και οι βασικές της αρχές θα αναλυθούν σε αυτό το κεφάλαιο.

