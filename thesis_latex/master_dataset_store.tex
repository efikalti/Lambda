\chapter{Master Dataset Store}

Αυτό το κεφάλαιο επικεντρώνετε στο σύστημα αποθήκευσης του master dataset που θα χρησιμοποιεί η αρχιτεκτονική Lambda. Η επιλογή ενός τέτοιου τρόπου θα επηρεάσει σε μεγάλο βαθμό την απόδοση όλου του συστήματος. Τα δεδομένα που αναμένετε να αποθηκευτούν απαιτούν πολύ περισσότερο αποθηκευτικό χώρο από αυτό που μπορεί να προσφέρει αποδοτικό ένας μόνο server. Συνεπώς απαιτείται η χρήση ενός κατανεμημένου συστήματος αποθήκευσης. Το σύστημα αποθήκευσης θα πρέπει επίσης να είναι σε θέση να χειριστεί τις συνεχείς προσθήκες στο master dataset χωρίς να απαιτεί πολλούς χειρισμούς από τους διαχειριστές.

\vspace{60mm}

\section{System Requirements}
Υπάρχουν τρεις απαιτήσεις που χρειάζεται να καλύπτει το σύστημα αποθήκευσης. Αυτές αναφέρονται στις πράξεις που θα πρέπει να γίνονται πάνω στα δεδομένα. 
Τα δεδομένα μας θα είναι immutable και eternal. Το οποίο σημαίνει ότι δεν θα υπάρχει κανένα Update πάνω στα υπάρχοντα δεδομένα. Το σύστημα θα πρέπει να υποστηρίζετε μόνο την Add και τη Delete σε πολύ συγκεκριμένες περιπτώσεις, απαιτήται επίσης να έχει πολύ καλή απόδοση στην εγγραφή καινούργιων δεδομένων σε μια μεγάλη συλλογή δεδομένων. Η πράξη της διαγραφής θεωρείται ότι θα συμβαίνει τόσο σπάνια που η απαίτηση να είναι optimal δεν είναι μεγάλης προτεραιότητας.
\newline  
Η δεύτερη πράξη που θα εκτελεί το σύστημα είναι η ανάγνωση των δεδομένων από τους αλγορίθμους που θα κατασκευάζουν τα Views στο Batch layer. Το σύστημα θα πρέπει να έχει πολύ καλή απόδοση στην ανάγνωση πολλών δεδομένων, ολόκληρου του dataset στην χειρότερη περίπτωση.
Τέτοια συστήματα περιγράφονται από το ακρωνύμιο WORM ( Write Once Read Many) και είναι συστήματα αποθήκευσης δεδομένων που αναπτύσσονται για την εξυπηρέτηση Big Data αλγόριθμων.
\newline
Οι απαιτήσεις φαίνονται πιο αναλυτικά στον παρακάτω πίνακα.

\begin{table}[!htb]
\centering
\caption{Απαιτήσεις Συστήματος}
\label{my-label}
\begin{tabular}{|M{2.5cm}|p{5cm}|p{8cm}|}
\hline
\rowcolor[HTML]{BBDAFF} 
                 Πράξη  & Απαιτήσεις & Σχόλια \\ \hline
                 Εγγραφή & Αποδοτική προσθήκη δεδομένων & Η μόνη πράξη εγγραφής εκτελείται για να προστεθούν δεδομένα στο master dataset. Δεν υπάρχει διαδικασία για Update. \\ \cline{2-3} 
                   & Scalable storage & Το master dataset, ανάλογα την περίπτωση και το θέμα της εφαρμογής, αναμένεται  να χρησιμοποιεί από πολλές δεκάδες Gigabytes έως Pettabytes για μεγάλες εφαρμογές. Θα πρέπει το σύστημα να μπορεί αποδοτικά να προσθέτει συνεχώς δεδομένα στο master dataset. \\ \hline
                 Ανάγνωση  & Υποστήριξη παράλληλης ανάγνωσης &  Η δημιουργία των Views στο Batch Layer θα γίνετε με την χρήση ολόκληρου του master dataset. Συνεπώς χρειάζεται υποστήριξη της παράλληλης επεξεργασίας για να μπορέσει να διαχειριστεί τον όγκο των δεδομένων αποδοτικά.\\ \hline
                 Εγγραφή και Ανάγνωση  & Ευελιξία στην διαχείριση του κόστους αποθήκευσης/επεξεργασίας & Η αποθήκευση τέτοιου μεγέθους δεδομένα είναι πολύ κοστοβόρα. Μία λύση αποτελεί η συμπίεση των δεδομένων για εξοικωνόμηση χώρου και η αποσυμπίεση τους για την επεξεργασία. Θα πρέπει να προσφέρετε η δυνατότητα συμπίεσης των δεδομένων και ανάκτησης τους όταν επιθυμεί ο χρήστης. \\ \cline{2-3} 
\multirow{-2}{*}{} & Αμεταβλητότητα δεδομένων & Θα πρέπει να ληφθούν όλοι οι δυνατοί έλεγχοι και συνθήκες για να τηρηθεί η αμεταβλητότητα του master dataset. \\ \hline
\end{tabular}
\end{table}
\clearpage

H τρίτη απαίτηση είναι εξίσου, αν όχι περισσότερο, σημαντική με τις προηγούμενες καθώς διασφαλίζει την ασφάλεια των δεδομένων και λαμβάνει όλες τις δυνατές ενέργειες για να μην υπάρχει καμία απώλεια. Βεβαίως κανένα σύστημα δεν μπορεί να εγγυηθεί ότι καλύπτει κάθε περίπτωση απώλειας δεδομένων με απόλυτη ακρίβεια. Η σωστή επιλογή ενός αρκετά αποδοτικού fault-tolerant συστήματος με συνεχείς ελέγχους και ένα μηχανισμό backup μπορεί να καλύψει το μεγαλύτερο κομμάτι των περιπτώσεων και να κρατήσει τα δεδομένα ασφαλή.


\section{Επιλογή συστήματος}
Σύμφωνα με αυτές τις απαιτήσεις θα γίνει η επιλογή του κατάλληλου κατανεμημένου συστήματος αποθήκευσης.Επειδή υπάρχουν πολλές κατηγορίες συστημάτων για αποθήκευση δεδομένων θα επιλεχθεί αεχικά ο τύπος του συστήματος και στη συνέχεια θα γίνει πιο εύκολη και η επιλογή ενός συγκεκριμένου.

\subsection{Σχεσιακές βάσεις δεδομένων}
Οι σχεσιακές βάσεις δεδομένων δεν ταιριάζουν ως σύστημα αποθήκευσης για τέτοιου είδους δεδομένα όπως έχει ήδη αναλυθεί. Χρειάζονται μεγάλο σχεδιασμό και χωρισμό των δεδομένων και των πεδίων τους κάτι το οποίο δεν είναι επιθυμητό. Επίσης οι βάσεις δεδομένων παρέχουν πολλούς μηχανισμούς για τροποποίηση των δεδομένων, το οποίο είναι άχρηστο και επιβλαβές στη συγκεκριμένη περίπτωση.

\subsection{Κey-value βάσεις δεδομένων}
Οι συγκεκριμένες βάσεις δεδομένων αποθηκεύουν τις πληροφορίες ως με έναν διαχωρισμό σε key και value, όπου key είναι ένα αναγνωριστικό για την μονάδα των πληροφοριών που αποθηκεύεται κάθε φορά, και value οι πληροφορίες. Ένα τέτοιο σύστημα μπορεί να χρησιμοποιηθεί από την αρχιτεκτονική για το master dataset όπως έχει περιγραφεί. Δεν προτείνετε για τους λόγους ότι περιέχει πολλές επιπλέον υπηρεσίες που δεν χρειάζονται, όπως indexing των δεδομένων, απαιτεί μια σχεδίαση για τον καταμερισμό των δεδομένων και παρέχει από μόνο του μηχανισμούς για Update οι οποίοι θα πρέπει να απενεργοποιηθούν με κάποιο τρόπο.

\subsection{Filesystem}
Οι βάσεις δεδομένων οποιασδήποτε μορφής απαιτούν μια εξοικείωση του χρήστη με τον τρόπο λειτουργίας, τους μηχανισμούς και το σχήμα της βάσης για να μπορέσει να τις χρησιμοποιήσει. Αντίθετα το Filesystem είναι ένα σύστημα αποθήκευσης με το οποίο όλοι είναι εξοικειωμένοι ως μέρος κάθε λειτουργικού συστήματος. Επίσης ένα Filesystem αποτελεί μια πολύ απλή λύση στο πρόβλημα της αποθήκευσης καθώς τα πάντα γράφονται σε αρχεία και αποθηκεύονται σε φακέλους. Οι μέθοδοι εγγραφής είναι εύκολοι και γρήγοροι, απλά επισυνάπτονται δεδομένα σε αρχεία ή δημιουργούνται νέα. Το μέγεθος των δεδομένων που αντιμετωπίζουμε όμως δεν μπορεί να το χειριστεί ένας υπολογιστής με τα μέσα που διαθέτει, συνεπώς η αρχιτεκτονική Lambda απαιτεί ένα Distributed Filesystem.  

\section{Distributed Filesystem}

Το πρόβλημα με το Filesystem, όπως έχει αναφερθεί είναι ότι ανήκει σε ένα μόνο μηχάνημα. Αυτό θέτει περιορισμούς στο μέγεθος του αποθηκευτικού χώρου και στην απόδοση του συστήματος. Τα προβλήματα αυτά έχουν ήδη επιλυθεί με την υλοποίηση των Distributed Filesystem, δηλαδή ενός Filesystem το οποίο αποτελείτε από πολλά διαφορικά μηχανήματα συνδεδεμένα μεταξύ του μέσω δικτύου. Οι λεπτομέρειες για το πως ακριβώς δουλεύει ένα τέτοιο σύστημα δεν είναι κομμάτι αυτής της εργασίας. Θα αναλυθούν όμως μία μία οι απαιτήσεις που προαναφέρθηκαν και πως τις καλύπτει το Distributed Filesystem.
\subsection{Immutability}
Οι ενέργειες που επιτρέπονται σε ένα Distributed Filesystem είναι σχεδόν ίδιες με αυτές ενός απλού Filesystem, με κάποιες μικρές διαφορές, μία από αυτές, η τροποποίηση αρχείων.Πολλές υλοποιήσεις των Distributed Filesystem απαγορεύουν την αλλαγή αρχείων που έχουν αποθηκευτεί. Ο μόνος τρόπος προσθήκης δεδομένων είναι με την δημιουργία και αποθήκευση νέων αρχείων. Συνεπώς αυτές οι υλοποιήσεις, έχουν ήδη διαφυλάξει την αμεταβλητότητα των αρχείων. Για κάθε εισαγωγή δεδομένων θα δημιουργείται και ένα νέο αρχείο.
\subsection{Scalability-Parallelism}
Τα αρχεία δεν είναι αποθηκευμένα σε ένα μηχάνημα, αλλά βρίσκονται διαχωρισμένα σε όλα τα μηχανήματα του συστήματος μας. Ο αποθηκευτικός χώρος είναι πολύ μεγαλύτερος από αυτόν που μπορεί να προσφέρει ένα μηχάνημα και η αύξηση του είναι πλέον πολύ απλή ενέργεια, με την εισαγωγή νέων μηχανημάτων. Επίσης τέτοια Distributed συστήματα παρέχουν έτοιμες παράλληλες τεχνικές για όλα τις πράξεις που πρέπει να εκτελεστούν ώστε να είναι αποδοτικά.
\subsection{Data Loss}
Τα Distributed Filesystem παρέχουν μηχανισμούς για την διαφύλαξη των δεδομένων από hardware failures, network failures κτλ. Το κάθε σύστημα έχει διαφορετικούς τρόπους για να το επιτύχει αλλά όλοι είναι αρκετά αποδοτικοί. Φυσικά κανένα σύστημα δεν μπορεί να είναι τελείως ασφαλές και να υπάρχει πρόβλεψη για κάθε πρόβλημα που μπορεί να προκύψει, αλλά με τα μέτρα που λαμβάνει το Distributed Filesystem, συνεχείς ελέγχους και περιοδικά backup του master dataset, η απώλεια δεδομένων αποτελέι πολύ σπάνια περίπτωση.
\subsection{Υλοποιήσεις}
Όπως αναφέρθηκε, υπάρχουν πολλές έτοιμες υλοποιήσεις Distributed Filesystem.
Κάποιες από αυτές, που παρέχουν και προστασία από απώλεια δεδομένων, ονομαστικά είναι:
\begin{itemize}
\item BeeGFS
\item Ceph
\item GFS (Google Inc.)
\item Windows Distributed File System (DFS) (Microsoft)
\item GlusterFS (Red Hat)
\item HDFS (Apache Software Foundation)
\item LizardFS (Skytechnology)
\item MooseFS (Core Technology / Gemius)
\item OneFS (EMC Isilon)
\item OrangeFS (Clemson University, Omnibond Systems)
\item ObjectiveFS
\item Panfs (Panasas)
\item RozoFS (Rozo Systems)
\item Parallel Virtual File System (Clemson University, Argonne National Laboratory, Ohio Supercomputer Center)
\item XtreemFS
\end{itemize}


\section{Σύνοψη}
Η επιλογή του σωστού συστήματος αποθήκευσης έχει επίσης μεγάλη συνεισφορά στην απόδοση του συστήματος. Η σωστή ανάλυση των απαιτήσεων και των συστημάτων που προσφέρονται αποτελεί ένα αναγκαίο βήμα στον σχεδιασμό της κάθε αρχιτεκτονικής. Οι κυριότερες απαιτήσεις που πρέπει να καλυφθούν αφοράν τα δεδομένα και την διαφύλαξη τους με κάθε δυνατό μέσο.
Άλλες απαιτήσεις που μπορούν να βοηθήσουν στην επιλογή του κατάλληλου συστήματος είναι η ευχέρεια και η γνώση των διαχειριστών με κάποιο/α από αυτά τα συστήματα, ο κώδικας του συστήματος και αν είναι open source, οι χρηματικοί πόροι. Αλλά αυτές οι απαιτήσεις είναι πολύ ειδικές και σχετίζονται με την κάθε εφαρμογή ξεχωριστά, συνεπώς δεν θα γίνει η ανάλυση τους.
\newline
Για τη συγκεκριμένη αρχιτεκτονική έχει επιλεχθεί το σύστημα HDFS.