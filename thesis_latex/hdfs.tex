\chapter{HDFS}
Το σύστημα HDFS (Hadoop Distributed File System) είναι ένα κατανεμημένο σύστημα αρχείων σχεδιασμένο για να λειτουργεί σε μεγάλο πλήθος μηχανημάτων. Έχει πολλές ομοιότητες με τα υπάρχοντα κατανεμημένα συστήματα αρχείων, αλλά και πολύ σημαντικές διαφορές.Το HDFS είναι εξαιρετικά ανεκτικό σε σφάλματα και έχει σχεδιαστεί για να λειτουργεί σε υλικό χαμηλού κόστους. Παρέχει υψηλό thoughput για τα δεδομένα της εφαρμογής και είναι κατάλληλο για εφαρμογές που έχουν μεγάλα σύνολα δεδομένων.Μερικές απαιτήσεις POSIX παραβλέπονται για να επιτραπεί και στις υπηρεσίες streaming τη πρόσβαση στα δεδομένα του συστήματος. Το HDFS αρχικά αναπτύχθηκε ως υποδομή για το έργο της μηχανής αναζήτησης Ιστού Apache Nutch. Τώρα είναι ένα κομμάτι του project Apache Hadoop.


\section{Αρχιτεκτονική}
H υλοποίηση του HDFS που θα χρησιμοποιηθεί είναι η έκδοση που περιλαμβάνετε στο Hadoop 2, το οποίο θα αναλυθεί στα επόμενα κεφάλαια. Ακολουθεί την αρχιτεκτονική Master-Slave που είναι και η αρχιτεκτονική ολόκληρου του Hadoop. Οι δύο κύριες υπηρεσίες του είναι το Namenode και το Datanode. Αυτές οι δύο αποτελούν τον κορμό του  συστήματος αλλά θα αναλυθούν και άλλες υπηρεσίες και ο ρόλος τους.

\subsection{Namenode}
Το Namenode είναι μια υπηρεσία master που εκτελείται σε ένα μηχάνημα του cluster και αποθηκεύει την κατάσταση των των αρχείων και καταλόγων σε μια ιεραρχία. Τα αρχεία αυτά περιέχουν στοιχεία όπως χαρακτηριστικά των αποθηκευμένων δεδομένων, όπως τα δικαιώματα, τροποποιήσεις και τους χρόνους πρόσβασης, ονομάτων και τον χώρο στο δίσκο. Τα δεδομένα που αποθηκεύονται στο HDFS είναι χωρισμένα σε blocks (συνήθως 128 megabytes, και κάθε block του αρχείου είναι αντιγραμένο σε πολλαπλά DataNodes. Το NameNode διατηρεί το namespace και την χαρτογράφηση του κάθε block στα DataNodes. Είναι μια πολύ σημαντική υπηρεσία χωρίς την οποία το σύστημα δεν μπορεί να λειτουργήσει. Για τον λόγο αυτό υπάρχει και μια υπηρεσία που ονομάζεται SecondaryNameNode η οποία τρέχει σε διαφορετικό μηχάνημα από το Namenode και κρατάει ένα πιστό όλων των δεδομένων του Namenode.Σε περίπτωση κατάρρευσης του Namenode, το σύστημα ειδοποιεί το SecondaryNamenode να αναλάβει την ευθύνη του Namenode μέχρι να επανέλθει σε σωστή λειτουργία.

\subsection{DataNode}
Η υπηρεσία Datanode εκτελείται στα slave μηχανήματα του cluster. Αναλαμβάνουν την αποθήκευση των blocks και των metadata που σχετίζονται με κάθε block.Τα metadata περιλαμβάνουν checksums για τον επιβεβαίωση της ακεραιότητας των δεδομένων και το generation stamp που δηλώνει την ακριβή ημερομηνία και ώρα εισαγωγής τους στο HDFS. Κατά την εκκίνηση, κάθε DataNode συνδέεται με το NameNode και εκτελεί ένα handshake. Ο σκοπός του είναι να εξακριβώσει την ταυτότητα του namespace και την έκδοση του λογισμικού του DataNode. Εάν είτε δεν ταιριάζει με αυτό του NameNode, η DataNode κλείνει αυτόματα.Μετά το handshake, το DataNode, εγγράφετε στον κατάλογο του NameNode και στέλνει ένα report με όλα τα blocks που κατέχει και κάποια metadata. Με αυτό τον τρόπο το Namenode γνωρίζει την τοποθεσία του κάθε block και σε πόσα διαφορετικά Datanodes βρίσκεται. Κατα την λειτουργία του Datanode, στέλνει σήματα στο Namenode ανα τακτά χρονικά διαστήματα, για να επιβεβαιώσει την σωστή λειτουργία του.Εάν κάποιο Datanode σταματήσει να στέλνει σήματα, θεωρείται εκτός λειτουργίας, το Namenode θα αναφέρει το σφάλμα και θα συνεχίσει την λειτουργία του με τα υπόλοιπα, τα blocks που βρίσκονται αποθηκευμένα σε αυτό το Datanode δεν είναι προσβάσιμα και αν ζητηθούν, θα χρησιμοποιηθούν τα αντίγραφα τους από άλλα Datanodes.

\subsection{HDFS Client}
Η πρόσβαση των εφαρμογών στα δεδομένα γίνεται μέσω του HDFS client. Είναι μια βιβλιοθήκη που παρέχει μεθόδους βασικά για την ανάκτηση και την εγγραφή δεδομένων από τον HDFS. Αυτή η υπηρεσία θα κάνει τις απαραίτητες αιτήσεις στο Namenode για κάποιο αρχείο ή μια λίστα με αρχεία. Δεν χρειάζετε να γνωρίζει το πως είναι διανεμημένα και αποθηκευμένα στον cluster, συνεπώς παρέχει ένα επίπεδο abstraction μεταξύ της εφαρμογής και της αρχιτεκτονικής του HDFS.
